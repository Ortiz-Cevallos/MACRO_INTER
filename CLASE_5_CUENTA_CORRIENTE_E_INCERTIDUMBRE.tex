\documentclass[10pt, xcolor=table, x11names]{beamer}
\usepackage[spanish]{babel} %CORTE DE PALABRAS RESPETANDO EL IDIOMA ESPAÑOL.
\usepackage[Utf8]{inputenc} %acentos desde el teclado
\usepackage	{textpos}
\usefonttheme{professionalfonts} % fuentes de LaTeX
\usetheme{Boadilla}      % or try Darmstadt, Madrid, Warsaw, ...
\usecolortheme[RGB={130,130,190}]{structure} % or try albatross, beaver, crane, ...
\useinnertheme{rounded}
%\useoutertheme{shadow}
\setbeamertemplate{blocks}[rounded][shadow=true]
\setbeamertemplate{navigation symbols}{}
\setbeamercovered{transparent} % Velos
\setbeamertemplate{caption}[numbered]
%\usepackage[spanish, authoryear, roud, datebegin]{flexbib} %CITAS BIBLIOGRÁFICAS
\newtheorem{Teorema}{Teorema}
\usepackage{ragged2e}
\justifying
\usepackage{booktabs}
\usepackage{multirow}
\usepackage[x11names,table]{xcolor}
%\usepackage[pdftex]{graphicx}
\usepackage{epstopdf} % Convertir .eps a .pdf (si fuera necesario)
\DeclareGraphicsExtensions{.pdf,.png,.jpg, .eps} % busca en este orden!
\author[Luis Ortiz Cevallos e-mail: \href{leortiz@uc.cl}{\textit{leortiz@uc.cl}}]{Profesor: Luis Ortiz Cevallos, e-mail:\href{leortiz@uc.cl}{\textit{leortiz@uc.cl}} }
\title[MACRO INTERNACIONAL]{\vspace*{1.0em} MACROECONOMÍA INTERNACIONAL}
\date[\href{https://ortiz-cevallos.github.io/luisortiz.github.io/ }{\textit{https://ortiz-cevallos.github.io/luisortiz.github.io/}}]{}
%\usepackage[pdftex]{hyperref}
\usepackage{tikz}
%\usepackage{pstricks}
\hypersetup{colorlinks,%
	citecolor=blue,%
	filecolor=blue,%
	linkcolor=blue,%
	urlcolor=blue,%
	pdftex}

\begin{document}



\begin{frame}
\titlepage
\end{frame}


\begin{frame}[label=1]
	\frametitle{{\normalsize CUENTA CORRIENTE EN UNA ECONOMÍA CON INCERTIDUMBRE } {}}
	\begin{block} {Motivación}
	Hay dos hechos estilizados en la economía EE.UU.
	\begin{description}
		\item[La gran moderación\footnote{Ver \cite{Quiros2000}, \cite{Kim1999} y \cite{Stock2002}}] \\
		
		\begin{itemize}
			\\
			\item Buena suerte
			\item Buenas políticas
			\item Cambio estructural 
		\end{itemize}
		 \item[Emerge un alto deterioro de la TB] 
	\end{description}	
	\end{block}	
	Ambos hechos ¿son coincidencia? o ¿cómo se explican? 
\end{frame}

\begin{frame}[label=2]
	\frametitle{{\normalsize UN MODELO CON INCERTIDUMBRE} {}}
	Dentro del marco del modelo visto anteriormente ahora asumimos que $Q_{2} $ no es conocido; se cree que éste puede ser alto o bajo en alguna probabilidad. \\
	
	Ante esa incertidumbre de manera intuitiva esperaríamos que surja el ahorro precautorio en el período 1, reduciendo en ese período el consumo y por ende mejorando la TB.\\
	
	Una conclusión sería que cuanto mayor sea la incertidumbre mejora la TB.   
	
	\begin{block} {Estructura}
		\begin{description}
			\item[Supuesto 1] Realizaciones del producto conocidas $Q_{1}=Q_{2}=Q $
			\item[Supuesto 2] Preferencias de la forma $ \ln{C_{1}}+ \ln{C_{2}}$ 
			\item[Supuesto 3] $ B_{0}^{*}=0$ y  $ r^{*}=0$ 
	\end{description}
	\end{block}	
\end{frame}

\begin{frame}[label=3]
	\frametitle{{\normalsize  UN MODELO CON INCERTIDUMBRE} {}}
	Bajo los supuestos anteriores la restricción de la economía es:
	\begin{align}
	C_{2}&=2Q-C_{1}
	\end{align}
	Lo que significa que el problema de los hogares es:
	\begin{align}
	\max_{C_{1}}& \ln{C_{1}}+ \ln{(2Q-C_{1})}\nonumber 
	 \end{align}
	 Cuya solución es $C_{1}=Q$	y por ende $C_{2}=Q $. Entonces el TB en el período 1 es:
	 \begin{align}
	 TB_{1}&=0
	 \end{align}
	 En conclusión en esta economía los hogares no necesitan ahorrar o des-ahorrar para estabilizar su consumo por que el ingreso ya está estabilizado.	
\end{frame}

\begin{frame}[label=4]
	\frametitle{{\normalsize  UN MODELO CON INCERTIDUMBRE} {}}
	Bajo los supuestos anteriores la restricción de la economía es:
	\begin{align}
	C_{2}&=2Q-C_{1}
	\end{align}
	Lo que significa que el problema de los hogares es:
	\begin{align}
	\max_{C_{1}}& \ln{C_{1}}+ \ln{(2Q-C_{1})}\nonumber 
	\end{align}
	Cuya solución es $C_{1}=Q$	y por ende $C_{2}=Q $. Entonces el TB en el período 1 es:
	\begin{align}
	TB_{1}&=0
	\end{align}
	En conclusión en esta economía los hogares no necesitan ahorrar o des-ahorrar para estabilizar su consumo por que el ingreso ya está estabilizado.	
\end{frame}

\begin{frame}[label=5]
	\frametitle{{\normalsize  UN MODELO CON INCERTIDUMBRE} {}}
	Ahora consideremos al situación en la que $Q_{2} $ no es conocido con certeza como  $Q_{1} $. Específicamente asumamos que que:
	\begin{align}
	Q_{2}&=\left\{ \begin{array}{lcl}
	Q+\sigma & \mbox{con probabilidad} & \frac{1}{2} \\
	 & &  \\
	Q-\sigma & \mbox{con probabilidad} & \frac{1}{2} 
	\end{array}
	\right.
	\end{align}
	Noten que el valor esperado de $Q_{2} $  es $Q $. Y que la desviación estándar de $Q_{2} $  es $ \sigma$\footnote{¿Por qué?}. En un cotexto con incertidumbre las preferencia de los hogares están dadas por:
	\begin{align}
	&\ln{C_{1}}+E\ln{C_{2}}
	\end{align}
	 Por tanto la restricción presupuestaría para el período 2 esta dado por:
	 \begin{align}
	 C_{2}&=\left\{ \begin{array}{lc}
	 2Q+\sigma-C_{1} & \mbox{en el estado bueno} \\
	 &   \\
	2Q-\sigma-C_{1} & \mbox{en el estado malo}
	 \end{array}
	 \right.
	 \end{align}	
\end{frame}

\begin{frame}[label=6]
	\frametitle{{\normalsize  UN MODELO CON INCERTIDUMBRE} {}}
Por tanto podemos definir la utilidad a lo largo de la vida de los hogares como:
\begin{align}
&\ln{C_{1}}+E\ln{C_{2}}\nonumber \\
&\ln{C_{1}}+\frac{1}{2}\ln{2Q+\sigma-C_{1}}+\frac{1}{2}\ln{2Q-\sigma-C_{1}}
\end{align}
Siendo por tanto el problema de de los hogares
\begin{align}
\max_{C_{1}}U&=\ln{C_{1}}+\frac{1}{2}\ln{2Q+\sigma-C_{1}}+\frac{1}{2}\ln{2Q-\sigma-C_{1}}\nonumber \\
\frac{\delta U}{\delta C_{1}}&=\frac{1}{C_{1}}-\frac{1}{2}\frac{1}{2Q+\sigma-C_{1}}-\frac{1}{2}\frac{1}{2Q-\sigma-C_{1}}\nonumber\\
0&=\frac{1}{C_{1}}-\frac{1}{2}(\frac{1}{2Q+\sigma-C_{1}}+\frac{1}{2Q-\sigma-C_{1}})\nonumber\\
\frac{1}{C_{1}}&=\frac{1}{2}(\frac{1}{2Q+\sigma-C_{1}}+\frac{1}{2Q-\sigma-C_{1}})
\end{align}

\end{frame}

\begin{frame}[label=7]
	\frametitle{{\normalsize  UN MODELO CON INCERTIDUMBRE} {}}
	La ecuación 9 nos indica que la utilidad marginal de consumir una unidad más en el período 1 debe ser igual a la utilidad marginal esperado del consumo de una unidad adicional en el período 2.\\
	Ahora pensemos que sucedería sí: $C_{1}=Q $; ello implicaría:
\begin{align}
\frac{1}{Q}&=\frac{1}{2}(\frac{1}{Q+\sigma}+\frac{1}{Q-\sigma})\nonumber \\
1&=\frac{1}{2}(\frac{Q}{Q+\sigma}+\frac{Q}{Q-\sigma})\nonumber \\
2&=\frac{Q}{Q+\sigma}+\frac{Q}{Q-\sigma}\nonumber \\
2&=\frac{Q(Q-\sigma)+Q(Q+\sigma)}{(Q+\sigma)(Q-\sigma)}\nonumber \\
2&=\frac{Q^{2}-Q\sigma+Q^{2}+Q\sigma}{Q^{2}-\sigma^{2}}\nonumber \\
1&=\frac{Q^{2}}{Q^{2}-\sigma^{2}}
\end{align}	
	
	
	
\end{frame}
\begin{frame}[label=8]
	\frametitle{{\normalsize  UN MODELO CON INCERTIDUMBRE} {}}
	Noten que 10 es imposible sí hay incertidumbre dado que $\sigma>0 $.\\
	Noten también que $\frac{1}{Q}<\frac{Q}{Q^{2}-\sigma^{2}} $ y por tanto 
	$U_{1}(C_{1},C_{2})<U_{2}(C_{1},C_{2}) $ y lo óptimo debe ser consumir en el período 1 menos que Q y por tanto consumir en el período 2 más que Q. Ello implica que $TB_{1}>0$\\
	
	\begin{block} {Conclusión}
		En un entorno con incertidumbre los hogares utilizan la balanza comercial como un vehículo para ahorrar más. Siendo este un comportamiento de ahorro precautorio\footnote{¿Por qué?}. Ello explica ambos hechos estilizados observados en EE.UU.
	\end{block}	
	
	  
	
	
\end{frame}

	\begin{frame}
		\frametitle{{\large 
				Bibliografía}}
		\renewcommand{\refname}{Referencias}
		\bibliography{Biblioteca}
		\bibliographystyle{flexbib}
	\end{frame}


\end{document}
\documentclass[10pt, xcolor=table, x11names]{beamer}
\usepackage[spanish]{babel} %CORTE DE PALABRAS RESPETANDO EL IDIOMA ESPAÑOL.
\usepackage[Utf8]{inputenc} %acentos desde el teclado
\usepackage	{textpos}
\usefonttheme{professionalfonts} % fuentes de LaTeX
\usetheme{Boadilla}      % or try Darmstadt, Madrid, Warsaw, ...
\usecolortheme[RGB={130,130,190}]{structure} % or try albatross, beaver, crane, ...
\useinnertheme{rounded}
%\useoutertheme{shadow}
\setbeamertemplate{blocks}[rounded][shadow=true]
\setbeamertemplate{navigation symbols}{}
\setbeamercovered{transparent} % Velos
\setbeamertemplate{caption}[numbered]
\usepackage[spanish, authoryear, roud, datebegin]{flexbib} %CITAS BIBLIOGRÁFICAS
\newtheorem{Teorema}{Teorema}
\usepackage{ragged2e}
\justifying
\usepackage{booktabs}
\usepackage{multirow}
\usepackage[x11names,table]{xcolor}
\usepackage[pdftex]{graphicx}
\usepackage{epstopdf} % Convertir .eps a .pdf (si fuera necesario)
\DeclareGraphicsExtensions{.pdf,.png,.jpg, .eps} % busca en este orden!
\title[AVERSIÓN AL RIESGO]{AVERSIÓN AL RIESGO}
\author[Luis Ortiz]{Luis Ortiz Cevallos}
\institute[U]{\bf UNIVERSIDAD}
\date[\today]{\footnotesize \today}
\usepackage[pdftex]{hyperref}
\usepackage{tikz}
\usepackage{pstricks}
\hypersetup{colorlinks,%
	citecolor=blue,%
	filecolor=blue,%
	linkcolor=blue,%
	urlcolor=blue,%
	pdftex}

\begin{document}


\begin{frame}
\titlepage
\end{frame}


\begin{frame}[label=1]
	\frametitle{{\normalsize COMPARANDO LA ADVERSIÓN AL RIESGO} {}}

Sean dos funciones de utilidades: $u_{1}$ y	$u_{2}$, ambas para un mismo nivel de riqueza,  $u_{1}$ corresponde a un sujeto con mayor aversión al riesgo que $u_{2}$. Si al sujeto 1 le desagrada todos los juegos de azar que le desagrade a 2 independientemente del nivel de riqueza inicial, implica que:\\
Dado \begin{equation}
\phi (x)=u_{1}(\frac{1}{u_{2}(x)})
\end{equation}
Sus propiedades son:
\begin{enumerate}
	\item $u_{1}(z)=\phi(u_{2}(z))$; en que $\phi(.) $ es una función que transforma $u_{2}$ en $u_{1}$.
	\item $u_{1}^{'}(z)=\phi^{'}(u_{2}(z))u_{2}^{'}(z)\rightarrow \phi^{'}=\frac{u_{1}^{'}}{u_{2}^{'}}>0$.
	\item
\end{enumerate}


\end{frame}


\begin{frame}[label=2]
	\frametitle{{\normalsize COMPARANDO LA ADVERSIÓN AL RIESGO} {}}
	
	\begin{align}
	u_{1}^{''}(z)&=\phi^{''}(u_{2}(z))(u_{2}^{'}(z))^{2}+\phi^{'}(u_{2}(z))u_{2}^{''}(z)\nonumber \\
	\phi^{''}(u_{2}^{'})^{2}&=u_{1}^{''}-\phi^{'}u_{2}^{''}\nonumber \\
	\phi^{''}&=\frac{u_{1}^{''}}{(u_{2}^{'})^{2}}-\frac{\phi^{'}u_{2}^{''}}{(u_{2}^{'})^{2}}\nonumber \\
	\phi^{''}&=\frac{u_{1}^{''}}{(u_{2}^{'})^{2}}+\frac{\phi^{'}}{u_{2}^{'}}A_{2}\nonumber \\
	\phi^{''}&=\frac{u_{1}^{''}}{(u_{2}^{'})^{2}}+\frac{u_{1}^{'}}{u_{2}^{'}}\frac{1}{u_{2}^{'}}A_{2}\nonumber \\
	\phi^{''}&=(\frac{u_{1}^{''}}{(u_{2}^{'})^{2}}+\frac{u_{1}^{'}}{u_{2}^{'}}\frac{1}{u_{2}^{'}}A_{2})\frac{1}{u_{1}^{'}}\nonumber\\
	\phi^{''}&=-A_{1}\frac{1}{(u_{2}^{'})^{2}}+\frac{1}{(u_{2}^{'})^{2}}A_{2}\nonumber  \\
	\phi^{''}&=\frac{1}{(u_{2}^{'})^{2}}(A_{2}-A_{1})<0\rightarrow A_{1}>A_{2} \nonumber  
	\end{align}
	
	
\end{frame}

\begin{frame}[label=3]
	\frametitle{{\normalsize COMPARANDO LA ADVERSIÓN AL RIESGO} {}}
	
	Una cuarta propiedad es que al considerar un riesgo $\tilde{x}$ que le sea desagradable a $u_{2}$ tenemos que:
	\begin{align}
	 Eu_{2}(w_{0}+\tilde{x})&\leq Eu_{2}(w_{0})\nonumber 
	 \end{align}
	Lo que implique que para el sujeto 1:
	\begin{align}
	Eu_{1}(w_{0}+\tilde{x})&= E\phi(u_{2}(w_{0}+\tilde{x}))\leq \phi(Eu_{2}(w_{0}+\tilde{x}))\leq\phi (u_{2}(w_{0}) )=u_{1}(w_{0})\nonumber 
	\end{align}
	
\end{frame}


\begin{frame}[label=4]
	\frametitle{{\normalsize COMPARANDO LA ADVERSIÓN AL RIESGO} {}}
	
	Una quinta propiedad se observa al incorporar una prima de riesgo $\pi$, la cual es la cantidad que uno está dispuesto a pagar para evitar un riesgo de media cero, de manera que:
	\begin{align}
       Eu(w_{0}+\tilde{x})=u(w_{0}-\pi)
	\end{align}
	Si definimos:
		\begin{align}
		z&=w_{0}-\pi\nonumber\\
		\tilde{y}&=\pi+\tilde{x}\nonumber
		\end{align}
Podemos reescribir 	2 como:
	\begin{align}
	Eu(z+\tilde{y})=u(z)
	\end{align}
Ahora si definimos $\pi_{2}$ como la prima de riesgo que paga el sujeto 2, y en base a ello definimos $\tilde{y}_{2}\; \wedge \;  z_{2}$ tenemos que:
\begin{align}
Eu_{2}(z_{2}+\tilde{y}_{2})=u_{2}(z_{2})\nonumber 
\end{align}	
Y como el sujeto 1 es más adverso al riesgo tenemos:
\begin{align}
Eu_{1}(z_{2}+\tilde{y}_{2})\leq u_{1}(z_{2})\nonumber 
\end{align}	
Noten que $z_{2}>z_{1}$ y por tanto $ \pi_{1}\geq \pi_{2}$.

\end{frame}
\end{document}
\documentclass[10pt, xcolor=table, x11names]{beamer}
\usepackage[spanish]{babel} %CORTE DE PALABRAS RESPETANDO EL IDIOMA ESPAÑOL.
\usepackage[Utf8]{inputenc} %acentos desde el teclado
\usepackage	{textpos}
\usefonttheme{professionalfonts} % fuentes de LaTeX
\usetheme{Boadilla}      % or try Darmstadt, Madrid, Warsaw, ...
\usecolortheme[RGB={130,130,190}]{structure} % or try albatross, beaver, crane, ...
\useinnertheme{rounded}
%\useoutertheme{shadow}
\setbeamertemplate{blocks}[rounded][shadow=true]
\setbeamertemplate{navigation symbols}{}
\setbeamercovered{transparent} % Velos
\setbeamertemplate{caption}[numbered]
%\usepackage[spanish, authoryear, roud, datebegin]{flexbib} %CITAS BIBLIOGRÁFICAS
\newtheorem{Teorema}{Teorema}
\usepackage{ragged2e}
\justifying
\usepackage{booktabs}
\usepackage{multirow}
\usepackage[x11names,table]{xcolor}
%\usepackage[pdftex]{graphicx}
\usepackage{epstopdf} % Convertir .eps a .pdf (si fuera necesario)
\DeclareGraphicsExtensions{.pdf,.png,.jpg, .eps} % busca en este orden!
\author[Luis Ortiz Cevallos e-mail: \href{leortiz@uc.cl}{\textit{leortiz@uc.cl}}]{Profesor: Luis Ortiz Cevallos, e-mail:\href{leortiz@uc.cl}{\textit{leortiz@uc.cl}} }
\title[MACRO INTERNACIONAL]{\vspace*{1.0em} MACROECONOMÍA INTERNACIONAL}
\date[\href{https://ortiz-cevallos.github.io/luisortiz.github.io/ }{\textit{https://ortiz-cevallos.github.io/luisortiz.github.io/}}]{}
%\usepackage[pdftex]{hyperref}
\usepackage{tikz}
%\usepackage{pstricks}
\hypersetup{colorlinks,%
	citecolor=blue,%
	filecolor=blue,%
	linkcolor=blue,%
	urlcolor=blue,%
	pdftex}

\begin{document}



\begin{frame}
\titlepage
\end{frame}


\begin{frame}[label=1]
	\frametitle{{\normalsize CUENTA CORRIENTE EN UNA ECONOMÍA CON PRODUCCIÓN } {}}
	\begin{block} {Objetivo}
		Extender la teoría sobre la determinación de la cuenta corriente en un economía con producción e inversión de capital físico. Por tanto la cuenta corriente será el resultado de ahorro menos inversión. 	
	\end{block}	
	\begin{block} {Estructura}
		\begin{description}
			\item[Supuesto 1] Se trata de una economía abierta con libre comercio de bienes y servicios y activos financieros con el resto del mundo.
			\item[Supuesto 2] Se trata de una economía pequeña. Ello significa que la tasa de interés internacional no depende de ninguna variable domestica.
			\item[Supuesto 3] Las personas en esa economía viven por dos períodos. 	
			\item[Supuesto 4] En esta economía hay dos sectores: Hogares y Firmas.
			\item[Supuesto 5] Las firmas producen un bien usando capital físico.
		\end{description}
	\end{block}	
\end{frame}

\begin{frame}[label=2]
	\frametitle{{\normalsize CUENTA CORRIENTE EN UNA ECONOMÍA CON PRODUCCIÓN } {}}
	En el período 1 las firmas demanda bienes de capital ($I_{1}$) para poder producir bienes finales en el período 2 ($Q_{2}$); formalmente:
		\begin{align}
		Q_{2}&=A_{2}F(I_{1})
		\end{align}
	Las propiedades de la función de producción son:
	
\begin{align}
		A_{2}F(0)&=0\nonumber\\
		A_{2}F^{'}(I_{1})&>0\nonumber\\
		A_{2}F^{''}(I_{1})&<0\nonumber
	
\end{align}	
\end{frame}



\begin{frame}[label=102]
	\frametitle{{\normalsize CUENTA CORRIENTE EN UNA ECONOMÍA CON PRODUCCIÓN } {}}
	\begin{center}
		\begin{tikzpicture}[scale=0.7, yscale=2]
		\draw[->] (0,0)  -- (8,0)  node[right] {$I_{1}$}; 
		\draw[->] (0,0)  -- (0,4)  node[left]  {$Q_{2}$};
		\draw[smooth, domain = 0:6, color=blue]
		plot (\x,{sqrt(\x)}) node[right] {$Q_{2}=AF(I_{1})$};
		\draw[fill=blue] (4,2) circle [radius=2.5pt]	node[above right] {};
		\draw[dotted, domain = 0:4, color=black]
		plot (\x,{2}); 
		\node[left] at (0,2) {$AF(I_{1}^{*})$};
		\draw[dotted, domain = 0:2, color=black]
		plot ({4},\x); 
		\node[below] at (4,0) {$I_{1}^{*}$};
		\draw[->, thick, black] (4.5,2.2360)  -- (5,2.2360);
		\node[left] at (4.5,2.2360) {{\footnotesize$slope=AF^{'}(I_{1})$}}; 
		\end{tikzpicture}
	\end{center}
\end{frame}


\begin{frame}[label=103]
	\frametitle{{\normalsize CUENTA CORRIENTE EN UNA ECONOMÍA CON PRODUCCIÓN } {}}
	\begin{center}
		\begin{tikzpicture}[scale=0.7, yscale=16]
		\draw[->] (0,0)  -- (8,0)  node[right] {$I_{1}$}; 
		\draw[->] (0,0)  -- (0,0.5)  node[left]  {$MPK$};
		\draw[smooth, domain = 1:6, color=blue]
		plot (\x,{pow(\x,-0.5)/2}) node[right] {$AF^{'}(I_{1})$};
		\draw[fill=blue] (4,0.25) circle node[above right] {};
		\draw[dotted, domain = 0:4, color=black]
		plot (\x,{0.25}); 
		\node[left] at (0,0.25) {$AF^{'}(I_{1}^{*})$};
		\draw[dotted, domain = 0:0.25, color=black]
		plot ({4},\x); 
		\node[below] at (4,0) {$I_{1}^{*}$};
		\end{tikzpicture}
	\end{center}
\end{frame}

\begin{frame}[label=3]
	\frametitle{{\normalsize CUENTA CORRIENTE EN UNA ECONOMÍA CON PRODUCCIÓN } {}}
	En el período 1 las firmas son deudoras en la economía para financiar su demanda de bienes de inversión: la deuda que poseen las firmas la denotamos como: $D_{1}^{f} $ y se cumple:
	\begin{align}
	D_{1}^{f}&=I_{1}
	\end{align}
	Las firmas prestan al interés $r_{1}$, por tanto en el período 2 deben pagar la deuda que asumieron con intereses. Se puede definir los beneficios de la firma como:
	\begin{align}
	\Pi_{2}&=A_{2}F(I_{1})-(1+r_{1})D_{1}^{f}\nonumber \\
	\Pi_{2}&=A_{2}F(I_{1})-(1+r_{1})I_{1}\label{e1}
	\end{align}
	
	
\end{frame}

\begin{frame}[label=4]
	\frametitle{{\normalsize CUENTA CORRIENTE EN UNA ECONOMÍA CON PRODUCCIÓN } {}}
	Noten que el problema de las firmas es:
		\begin{align}
		\max_{I_{1}}\Pi_{2}&=A_{2}F(I_{1})-(1+r_{1})I_{1}\nonumber
		\end{align}
	Tomando como dado el resto de argumentos: ($A_{2}, r_{1} $). Cada unidad de inversión le cuesta a la firma $1+r_{1} $ en el período 2 (esto es el costo del capital). Así que para niveles de inversión cercano a cero el producto marginal de una unidad adicional de inversión es superior a su costo y por tanto la firma tiene incentivo de aumentarla. Por tanto el equilibrio cumple con:
	\begin{align}
	1+r_{1}&=A_{2}F^{'}(I_{1})
	\end{align}
\end{frame}


\begin{frame}[label=104]
	\frametitle{{\normalsize CUENTA CORRIENTE EN UNA ECONOMÍA CON PRODUCCIÓN } {}}
	\begin{center}
		\begin{tikzpicture}[scale=0.7, yscale=16]
		\draw[->] (0,0)  -- (8,0)  node[right] {$I_{1}$}; 
		\draw[->] (0,0)  -- (0,0.5)  node[left]  {$MPK$};
		\draw[smooth, domain = 1:6, color=blue]
		plot (\x,{pow(\x,-0.5)/2}) node[right] {$AF^{'}(I_{1})$};
		\draw[smooth, domain = 0:6, color=blue]
		plot (\x,{0.25}) node[right] {$ $};
		\draw[fill=blue] (4,0.25) circle node[above right] {};
		\draw[dotted, domain = 0:0.25, color=blue]
		plot ({4},\x); 
		\node[below] at (4,0) {$I_{1}^{*}$};
		\node[left] at (0,0.25) {$1+r_{1}$}
		\end{tikzpicture}
	\end{center}
\end{frame}

\begin{frame}[label=105]
	\frametitle{{\normalsize CUENTA CORRIENTE EN UNA ECONOMÍA CON PRODUCCIÓN } {\small INCREMENTO DE TASAS}}
	\begin{center}
		\begin{tikzpicture}[scale=0.7, yscale=16]
		\draw[->] (0,0)  -- (8,0)  node[right] {$I_{1}$}; 
		\draw[->] (0,0)  -- (0,0.5)  node[left]  {$MPK$};
		\draw[smooth, domain = 1:6, color=blue]
		plot (\x,{pow(\x,-0.5)/2}) node[right] {$AF^{'}(I_{1})$};
		\draw[smooth, domain = 0:6, color=blue]
		plot (\x,{0.25}) node[left] {$ $};
		\draw[smooth, domain = 0:6, color=red]
		plot (\x,{0.30}) node[right] {};
		\draw[fill=blue] (4,0.25) circle node[above right] {};
		\draw[dotted, domain = 0:0.25, color=blue]
		plot ({4},\x); 
		\draw[dotted, domain = 0:0.3, color=red]
		plot ({2.7777},\x); 
		\node[below] at (4,0) {$I_{1}^{*}$};
		\node[below] at (2.7777,0) {$I_{1}^{**}$};
		\node[left] at (0,0.25) {$1+r_{1}$};
		\node[left] at (0,0.30) {$1+r_{1}^{'}$}
		\end{tikzpicture}
	\end{center}
\end{frame}

\begin{frame}[label=106]
	\frametitle{{\normalsize CUENTA CORRIENTE EN UNA ECONOMÍA CON PRODUCCIÓN } {}}
	\begin{center}
		\begin{tikzpicture}[scale=0.7, yscale=16]
		\draw[->] (0,0)  -- (8,0)  node[right] {$I_{1}$}; 
		\draw[->] (0,0)  -- (0,0.5)  node[left]  {$r_{1}$};
		\draw[smooth, domain = 1:6, color=black]
		plot (\x,{pow(\x,-0.5)/2}) node[right] {$I(r_{1})$};
		
		\end{tikzpicture}
	\end{center}
\end{frame}

\begin{frame}[label=5]
	\frametitle{{\normalsize CUENTA CORRIENTE EN UNA ECONOMÍA CON PRODUCCIÓN } {}}
	En resumen podemos definir  el nivel de inversión de una economía como:
	\begin{equation}
	I_{1}=I(A_{2}; r_{1}) \atop \hspace{1.0cm}+\;-
	\end{equation}
	Tomen en cuenta que no se ha dicho nada sobre los beneficios de las firmas en el período 1. ¿Por qué?
\end{frame}


\begin{frame}[label=6]
	\frametitle{{\normalsize CUENTA CORRIENTE EN UNA ECONOMÍA CON PRODUCCIÓN } {Hogares}}
	En el período 1 los hogares están dotados por un stock de activos: $B_{0}^{h} $ los cuales devengan un flujo de intereses de $ r_{0}B_{0}^{h}$ donde $r_{0}$ es el interés por los activos mantenidos del período 0 al 1.\\
	Asumimos que los hogares son propietarios de las firmas. Ello implica que sus ingresos en el periodo 1 incluyen los beneficios de las firmas: $\Pi_{1}$.\\
	Los hogares usan sus ingresos en consumo e incremento de activos (ahorro); por tanto su restricción presupuestaria en el período 1 es:
	 \begin{align}
	 C_{1}+(B_{1}^{h}-B_{0}^{h})&=\Pi_{1}+r_{0}B_{0}^{h}
	 \end{align}
	 Similarmente en el período 2 es:
	 \begin{align}
	 C_{2}+(B_{2}^{h}-B_{1}^{h})&=\Pi_{2}+r_{1}B_{1}^{h}\nonumber \\
	 C_{2}-B_{1}^{h}&=\Pi_{2}+r_{1}B_{1}^{h}\nonumber \\
	 C_{2}&=\Pi_{2}+(1+r_{1})B_{1}^{h}
	 \end{align}

\end{frame}

\begin{frame}[label=7]
	\frametitle{{\normalsize CUENTA CORRIENTE EN UNA ECONOMÍA CON PRODUCCIÓN } {Hogares}}
	Combinando las ecuaciones 6 y 7 tenemos la restricción presupuestaria intertemporal:
	\begin{align}
	 C_{2}&=\Pi_{2}+(1+r_{1})B_{1}^{h}\nonumber \\
	C_{2}&=\Pi_{2}+(1+r_{1})(\Pi_{1}+(1+r_{0})B_{0}^{h}-C_{1})\nonumber \\
	\frac{C_{2}}{(1+r_{1})}+C_{1}&=\frac{\Pi_{2}}{(1+r_{1})}+\Pi_{1}+(1+r_{0})B_{0}^{h}
	\end{align}
	Los hogares enfrentan el siguiente problema ínter-temporal:
	\begin{align}
	\max_{C_{1}, C_{2}} U(C_{1}, C_{2})\nonumber \\
	s.a.\nonumber \\
	\frac{C_{2}}{(1+r_{1})}+C_{1}&=\frac{\Pi_{2}}{(1+r_{1})}+\Pi_{1} +(1+r_{0})B_{0}^{*}\nonumber \\
	C.P.O\nonumber \\
	U_{1}(C_{1}, C_{2})&=(1+r_{1})U_{2}(C_{1}, C_{2}) 
	\end{align}
\end{frame}

\begin{frame}[label=8]
	\frametitle{{\normalsize CUENTA CORRIENTE EN UNA ECONOMÍA CON PRODUCCIÓN } {Equilibrio}}
	En el caso de una economía pequeña y abierta con movilidad de capitales los hogares y firman prestan y toman prestado a una tasa exógenos mundial ($ r^{*}$) de manera que se cumple:
	\begin{align}
		r_{1}&=r^{*}
	\end{align}
	La posición de inversión internacional neta de la economía ($B_{0}^{*} $) se define como:
		\begin{align}
			B_{0}^{*}&=B_{0}^{h}-D_{0}^{f}
		\end{align}
	Si sustituimos en 8 los beneficios ($\Pi $) y luego reemplazamos 11 considerando que $D_{1}^{f}=I_{1} $ tenemos:
	 \begin{align}
	 	\frac{C_{2}}{(1+r_{1})}+C_{1}&=\frac{\Pi_{2}}{(1+r_{1})}+\Pi_{1} +(1+r_{0})B_{0}^{*}\nonumber \\
	 \frac{C_{2}}{(1+r_{1})}+C_{1}&=\frac{(A_{2}F(I_{1})-(1+r_{1})D_{1}^{f})}{(1+r_{1})}+(A_{1}F(I_{0})-(1+r_{0})D_{0}^{f}) \nonumber \\
	 &+(1+r_{0})B_{0}^{*}\nonumber \\
	 \frac{C_{2}}{(1+r_{1})}+C_{1}+I_{1}&=\frac{A_{2}F(I_{1})}{(1+r_{1})}+A_{1}F(I_{0})+(1+r_{0})B_{0}^{h}
	 \end{align}
\end{frame}

\begin{frame}[label=8]
	\frametitle{{\normalsize CUENTA CORRIENTE EN UNA ECONOMÍA CON PRODUCCIÓN } {Equilibrio}}
	Noten que de la ecuación 12 el lado izquierdo denota la absorción de la economía en los dos períodos y el lado derecho denota el valor de la riqueza de la economía en los dos período; ambos concepto traído a valor presente del período 1.
	En resumen el equilibrio es la asignación ${C_{1}, C_{2}, I_{1} \; y\; r_{1}} $ que satisfagan:
	 \begin{align}
	 r_{1}&=r^{*}\nonumber \\
	 U_{1}(C_{1}, C_{2})&=(1+r^{*})U_{2}(C_{1}, C_{2})\nonumber \\
	 1+r^{*}&=A_{2}F^{'}(I_{1})\nonumber \\
	 \frac{C_{2}}{(1+r^{*})}+C_{1}+I_{1}&=\frac{A_{2}F(I_{1})}{(1+r^{*})}+A_{1}F(I_{0})+(1+r_{0})B_{0}^{h}\nonumber
	 \end{align}
Dado los valores de riqueza inicial, tasa de interés inicial y extranjera y los niveles de productividad de los períodos 1 y 2.
Sí definimos el producto 1 como:
\begin{align}
Q_{1}&=A_{1}F(I_{0})\nonumber 
\end{align}
Noten que $A_{1} $ es determinado en el período 1 pero es exógeno para las firmas; y que $I_{0} $ está predeterminado. Por tanto $Q_{1} $ se determina en 1 pero es exógeno. 
\end{frame}

\begin{frame}[label=8]
	\frametitle{{\normalsize CUENTA CORRIENTE EN UNA ECONOMÍA CON PRODUCCIÓN } {Conclusión }}
	En base a lo anterior podemos definir la balanza comercial para el período 1:
	\begin{align}
	TB_{1}&=Q_{1}-C_{1}-I_{1} 
	\end{align}
	Sí el producto en 2 están dado por:
	\begin{align}
	Q_{2}&=A_{2}F(I_{2})\nonumber 
	\end{align}
Este producto es una variable endógena en el modelo siendo la balanza comercia en dos dado por:
\begin{align}
TB_{2}&=Q_{2}-C_{2} 
\end{align}
Las cuenta corriente se definen para el período 1 y 2 como:
\begin{align}
CA_{1}&=TB_{1}-r_{0}B_{0}^{*} \\
CA_{1}&=B_{1}^{*}-B_{0}^{*} \nonumber \\
CA_{2}&=TB_{2}-r^{*}B_{1}^{*} \\
CA_{2}&=-B_{1}^{*} \nonumber
\end{align}
Finalmente el ahorro en la economía es:
\begin{align}
S_{1}&=Q_{1}+r_{0}B_{0}^{*}-C_{1} \\
\end{align}

\end{frame}



\end{document}
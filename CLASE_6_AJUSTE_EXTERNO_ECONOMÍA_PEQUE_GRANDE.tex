\documentclass[10pt, xcolor=table, x11names]{beamer}
\usepackage[spanish]{babel} %CORTE DE PALABRAS RESPETANDO EL IDIOMA ESPAÑOL.
\usepackage[Utf8]{inputenc} %acentos desde el teclado
\usepackage	{textpos}
\usefonttheme{professionalfonts} % fuentes de LaTeX
\usetheme{Boadilla}      % or try Darmstadt, Madrid, Warsaw, ...
\usecolortheme[RGB={130,130,190}]{structure} % or try albatross, beaver, crane, ...
\useinnertheme{rounded}
%\useoutertheme{shadow}
\setbeamertemplate{blocks}[rounded][shadow=true]
\setbeamertemplate{navigation symbols}{}
\setbeamercovered{transparent} % Velos
\setbeamertemplate{caption}[numbered]
%\usepackage[spanish, authoryear, roud, datebegin]{flexbib} %CITAS BIBLIOGRÁFICAS
\newtheorem{Teorema}{Teorema}
\usepackage{ragged2e}
\justifying
\usepackage{booktabs}
\usepackage{multirow}
\usepackage[x11names,table]{xcolor}
%\usepackage[pdftex]{graphicx}
\usepackage{epstopdf} % Convertir .eps a .pdf (si fuera necesario)
\DeclareGraphicsExtensions{.pdf,.png,.jpg, .eps} % busca en este orden!
\author[Luis Ortiz Cevallos e-mail: \href{leortiz@uc.cl}{\textit{leortiz@uc.cl}}]{Profesor: Luis Ortiz Cevallos, e-mail:\href{leortiz@uc.cl}{\textit{leortiz@uc.cl}} }
\title[MACRO INTERNACIONAL]{\vspace*{1.0em} MACROECONOMÍA INTERNACIONAL}
\date[\href{https://ortiz-cevallos.github.io/luisortiz.github.io/ }{\textit{https://ortiz-cevallos.github.io/luisortiz.github.io/}}]{}
%\usepackage[pdftex]{hyperref}
\usepackage{tikz}
%\usepackage{pstricks}
\hypersetup{colorlinks,%
	citecolor=blue,%
	filecolor=blue,%
	linkcolor=blue,%
	urlcolor=blue,%
	pdftex}

\begin{document}


\begin{frame}
\titlepage
\end{frame}


\begin{frame}[label=11]
	\frametitle{{\normalsize AJUSTE EXTERNO } {}}
	\begin{block} {Motivación}
		Presentar una herramienta gráfica que ayude a entender los ajustes externos de una economía.	
	\end{block}	
	La figura 1 resumen el comportamiento del ahorro e inversión y por tanto determina la CA. Sus componentes son:
	\begin{enumerate}
		\item La inversión como una relación negativa de la tasa de interés, dado el costo del capital que asumen las firmas ante el problema de maximizar beneficios.
		\item El ahorro el cual depende de manera positiva tanto de la tasa de interés (a través de los canales: sustitución, ingreso y beneficios futuros) como del producto (efecto transitorio).
		\item la CA queda por tanto determinada como un función positiva de la tasa de interés y el producto
	\end{enumerate} 
\end{frame}

\begin{frame}[label=1]
	\frametitle{{\normalsize TEORÍA SOBRE DETERMINACIÓN DE LA CUENTA CORRIENTE} {}}
	\begin{figure}[H]
		\renewcommand{\figurename}{Figura}
		\caption{\sc Ahorro, Inversión y Cuenta Corriente}
		\begin{minipage}[t]{0.48\textwidth}
		\begin{center}
			\begin{tikzpicture}[scale=0.5]
			\draw[-] (0,0)  -- (8,0)  node[right] {$S,I$}; 
			\draw[-] (0,0)  -- (0,8)  node[left]  {$r_{1}$};
			\draw[smooth, domain = 0.25:6.5, color=black]
			plot (\x,{7-\x}) node[right] {$I(r_{1}) $};
			\draw[smooth, domain = 0.25:6, color=black]
			plot (\x,{0+\x}) node[right] {$S(r_{1},Q) $};
			\draw[dotted, domain = 0:8, color=black]
			plot (\x,{3.5}); 
			\node[left] at (0,3.5) {$r^{c}$};
			\draw[dotted, domain = 0:8, color=black]
			plot (\x,{5}); 
			\node[left] at (0,5) {$r^{a}$};
			\draw[dotted, domain = 0:8, color=black]
			plot (\x,{1}); 
			\node[left] at (0,1) {$r^{b}$};
			\end{tikzpicture}
		\end{center}		
		\end{minipage} \hfill \begin{minipage}[t]{0.48\textwidth}
		\begin{center}
			\begin{tikzpicture}[scale=0.5]
			\draw[-] (-4,0)  -- (4,0)  node[right] {$CA$}; 
			\draw[-] (0,0)  -- (0,8)  node[left]  {$r_{1}$};
			\draw[smooth, domain = -3:3, color=black]
			plot (\x,{3.5+\x}) node[right] {$CA(r_{1}) $};
			\draw[dotted, domain = 0:-4, color=black]
			plot (\x,{3.5}); 
			\node[left] at (0,3.5) {$r^{c}$};
			\draw[dotted, domain = -4:1.5, color=black]
			plot (\x,{5}); 
			\node[left] at (0,5) {$r^{a}$};
			\draw[dotted, domain = 0:5, color=black]
			plot (1.5,{\x}); 
			\draw[dotted, domain = -4:-2.5, color=black]
			plot (\x,{1}); 
			\node[left] at (0,1) {$r^{b}$};
			\draw[dotted, domain = 1:0, color=black]
			plot (-2.5,{\x}); 
			
			\end{tikzpicture}
		\end{center}
	\end{minipage}
	\label{IV_G3}
\end{figure}
\end{frame}

\begin{frame}[label=12]
	\frametitle{{\normalsize AJUSTE EXTERNO ECONOMÍA PEQUEÑA} {}}
	En una economía pequeña y abierta debe cumplirse que:
	\begin{align}
	r_{1}&=r^{*}
	\end{align}
	Lo que implica que la CA en la herramienta gráfica dado por el plano r.CA debe de ser aquella en que se la CA(r,Q) se intercepta con la horizontal $r^{*}$ dado Q.
	 
\end{frame}

\begin{frame}[label=2]
	\frametitle{{\normalsize AJUSTE EXTERNO ECONOMÍA PEQUEÑA} {}}
	\begin{figure}[H]
		\renewcommand{\figurename}{Figura}
		\caption{\sc Condición de ajuste externo economía pequeña}
	\begin{center}
		\begin{tikzpicture}[scale=0.8]
		\draw[-] (-4,0)  -- (4,0)  node[right] {$CA$}; 
		\draw[-] (0,0)  -- (0,8)  node[left]  {$r_{1}$};
		\draw[smooth, domain = -3:3, color=black]
		plot (\x,{3.5+\x}) node[right] {$CA(r_{1}) $};
		\draw[smooth, domain = -4:4, color=blue]
		plot (\x,{2}) node[right] {$r^{*} $};
		\draw[dotted, domain = 2:0, color=black]
		plot (-1.5,{\x}); 
		\node[below] at (-1.5,0) {$CA(r^{*})$};
		\draw[fill] (-1.5,2) circle [radius=1.5pt];
	\end{tikzpicture}
	\end{center}
	\end{figure}
\end{frame}



\begin{frame}[label=3]
	\frametitle{{\normalsize AJUSTE EXTERNO SHOCK EN TASA DE INTERÉS } {}}
	\begin{center}
		\begin{tikzpicture}[scale=0.8]
		\draw[-] (-4,0)  -- (4,0)  node[right] {$CA$}; 
		\draw[-] (0,0)  -- (0,8)  node[left]  {$r_{1}$};
		\draw[smooth, domain = -3:3, color=black]
		plot (\x,{3.5+\x}) node[right] {$CA(r_{1},Q_{1}) $};
		\draw[dotted, domain = -4:4, color=black]
		plot (\x,{3}) node[right] {$r^{*1} $};
		\draw[dotted, domain = -4:4, color=black]
		plot (\x,{1}) node[right] {$r^{*0} $};
		\draw[dotted, domain = 1:0, color=black]
		plot (-2.5,{\x}); 
		\node[below] at (-2.5,0) {$CA^{0}$};
		\draw[dotted, domain = 3:0, color=black]
		plot (-0.5,{\x}); 
		\node[below] at (-0.5,0) {$CA^{1}$};
		\end{tikzpicture}
	\end{center}
\end{frame}


\begin{frame}[label=4]
	\frametitle{{\normalsize AJUSTE EXTERNO SHOCK DEL PRODUCTO TRANSITORIO} {}}
	\begin{figure}[H]
		\renewcommand{\figurename}{Figura}
		\caption{\sc Ahorro, Inversión y Cuenta Corriente}
		\begin{minipage}[t]{0.48\textwidth}
			\begin{center}
				\begin{tikzpicture}[scale=0.48]
				\draw[-] (0,0)  -- (8,0)  node[right] {{\tiny$S,I$}}; 
				\draw[-] (0,0)  -- (0,8)  node[left]  {{\tiny$r_{1}$}};
				\draw[smooth, domain = 0.25:6.5, color=black]
				plot (\x,{7-\x}) node[right] {{\tiny$I(r_{1}) $}};
				\draw[smooth, domain = 0.25:7, color=black]
				plot (\x,{0+\x}) node[right] {{\tiny$S(r_{1},Q_{1}^{0}) $}};
				\draw[smooth, domain = 2.25:8, color=black]
				plot (\x,{-2+\x}) node[right] {{\tiny$S(r_{1},Q_{1}^{1}) $}};
				\draw[dotted, domain = 0:4.5, color=black]
				plot (\x,{2.5}); 
				\node[left] at (0,3.5) {$r_{c}^{0}$};
				\draw[dotted, domain = 0:3.5, color=black]
				plot (\x,{3.5}); 
				\node[left] at (0,2.5) {$r_{c}^{1}$};
				\draw[dotted, domain = 0:8, color=black]
				plot (\x,{1}); 
				\node[left] at (0,1) {$r^{*}$};
				\draw[dotted, domain = 1:0, color=black]
				plot (1,{\x}); 
				\node[below] at (1,0) {{\tiny$S_{1}^{0}$}};
				\draw[dotted, domain = 1:0, color=black]
				plot (3,{\x}); 
				\node[below] at (3,0) {{\tiny$S_{1}^{1}$}};
				\draw[dotted, domain = 1:0, color=black]
				plot (6,{\x}); 
				\node[below] at (6,0) {{\tiny$I_{1}^{0}$}};
				\end{tikzpicture}
			\end{center}		
		\end{minipage} \hfill \begin{minipage}[t]{0.48\textwidth}
		\begin{center}
			\begin{tikzpicture}[scale=0.48]
			\draw[-] (-4,0)  -- (4,0)  node[right] {{\tiny $CA$}}; 
			\draw[-] (0,0)  -- (0,8)  node[left]  {{\tiny$r_{1}$}};
			\draw[smooth, domain = -3:4, color=black]
			plot (\x,{3.5+\x}) node[right] {{\tiny $CA(r_{1},Q_{1}^{0}) $}};
			\draw[smooth, domain = -1:4, color=black]
			plot (\x,{1.5+\x}) node[right] {{\tiny $CA(r_{1},Q_{1}^{1}) $}};
			\draw[dotted, domain = -4:4, color=black]
			plot (\x,{1}); 
			\draw[dotted, domain = 1:0, color=black]
			plot (-2.5,{\x}); 
			\node[below] at (-2.5,0) {{\tiny$CA_{1}^{0}$}};
			\draw[dotted, domain = 1:0, color=black]
			plot (-0.5,{\x}); 
			\node[below] at (-0.5,0) {{\tiny$CA_{1}^{1}$}};
			\end{tikzpicture}
		\end{center}
	\end{minipage}
	\label{IV_G2}
\end{figure}
\end{frame}

\begin{frame}[label=5]
	\frametitle{{\normalsize AJUSTE EXTERNO SURGIMIENTO DE INVERSIÓN} {}}
	\begin{figure}[H]
		\renewcommand{\figurename}{Figura}
		\caption{\sc Ahorro, Inversión y Cuenta Corriente}
		\begin{minipage}[t]{0.48\textwidth}
			\begin{center}
				\begin{tikzpicture}[scale=0.48]
				\draw[-] (0,0)  -- (8,0)  node[right] {{\tiny$S,I$}}; 
				\draw[-] (0,0)  -- (0,8)  node[left]  {{\tiny$r_{1}$}};
				\draw[smooth, domain = 6.5:0.25, color=black]
				plot (\x,{7-\x}) node[right] {{\tiny$I^{0}(r_{1}) $}};
				\draw[smooth, domain = 7.5:2.25, color=blue]
				plot (\x,{8-\x}) node[right] {{\tiny$I^{1}(r_{1}) $}};
				\draw[smooth, domain = 0.25:6, color=black]
				plot (\x,{0+\x}) node[right] {{\tiny$S^{0}(r_{1},Q_{1}) $}};
				\draw[smooth, domain = 0.75:6.75, color=blue]
				plot (\x,{0.75+\x}) node[right] {{\tiny$S^{1}(r_{1},Q_{1}) $}};
				\draw[dotted, domain = 0:3.5, color=black]
				plot (\x,{3.5}); 
				\node[left] at (0,3.5) {{\tiny$r_{c}^{0}$}};
				\draw[dotted, domain = 0:3.625, color=black]
				plot (\x,{4.375}); 
				\node[left] at (0,4.375) {{\tiny$r_{c}^{1}$}};
				\draw[dotted, domain = 0:8, color=black]
				plot (\x,{2}); 
				\node[left] at (0,2) {$r^{*}$};
				\draw[dotted, domain = 2:0, color=black]
				plot (1.25,{\x}); 
				\node[below] at (1.25,0) {{\tiny$S_{1}^{1}$}};
				\draw[dotted, domain = 2:0, color=black]
				plot (2,{\x}); 
				\node[below] at (2,0) {{\tiny$S_{1}^{0}$}};
				\draw[dotted, domain = 2:0, color=black]
				plot (5,{\x}); 
				\node[below] at (5,0) {{\tiny$I_{1}^{0}$}};
				\draw[dotted, domain = 2:0, color=black]
				plot (6,{\x}); 
				\node[below] at (6,0) {{\tiny$I_{1}^{1}$}};
				\end{tikzpicture}
			\end{center}		
		\end{minipage} \hfill \begin{minipage}[t]{0.48\textwidth}
		\begin{center}
			\begin{tikzpicture}[scale=0.48]
			\draw[-] (-4,0)  -- (4,0)  node[right] {{\tiny $CA$}}; 
			\draw[-] (0,0)  -- (0,8)  node[left]  {{\tiny$r_{1}$}};
			\draw[smooth, domain = -3:3, color=black]
			plot (\x,{3.5+\x}) node[right] {{\tiny $CA^{0}(r_{1},Q_{1}) $}};
			\draw[smooth, domain = -4:3, color=blue]
			plot (\x,{4.5+\x}) node[right] {{\tiny $CA^{1}(r_{1},Q_{1}) $}};
			\draw[dotted, domain = -4:4, color=black]
			plot (\x,{2}); 
			\draw[dotted, domain = 2:0, color=black]
			plot (-1.5,{\x}); 
			\node[below] at (-1.5,0) {{\tiny$CA_{1}^{0}$}};
			\draw[dotted, domain = 2:0, color=black]
			plot (-2.5,{\x}); 
			\node[below] at (-2.5,0) {{\tiny$CA_{1}^{1}$}};
			\end{tikzpicture}
		\end{center}
	\end{minipage}
	\label{IV_G4}
\end{figure}
\end{frame}

\begin{frame}[label=6]
	\frametitle{{\normalsize AJUSTE EXTERNO PREMIO POR RISGO PAÍS} {}}
	\begin{center}
		\begin{tikzpicture}[scale=0.8]
		\draw[-] (-4,0)  -- (4,0)  node[right] {$CA$}; 
		\draw[-] (0,0)  -- (0,8)  node[left]  {$r_{1}$};
		\draw[smooth, domain = -3:3, color=black]
		plot (\x,{3.5+\x}) node[right] {$CA(r_{1},Q_{1}) $};
		\draw[dotted, domain = -4:0, color=black]
		plot (\x,{3}) node[right] {$r^{*}+\rho  $};
		\draw[dotted, domain = -4:0, color=black]
		plot (\x,{1}) node[right] {$r^{*} $};
		\draw[dotted, domain = 1:0, color=black]
		plot (-2.5,{\x}); 
		\node[below] at (-2.5,0) {$ $};
		\draw[dotted, domain = 3:0, color=black]
		plot (-0.5,{\x}); 
		\node[below] at (-0.5,0) {$CA^{0}$};
		\end{tikzpicture}
	\end{center}
\end{frame}

\begin{frame}[label=7]
	\frametitle{{\normalsize AJUSTE EXTERNO PREMIO POR RISGO PAÍS} {}}
	\begin{center}
		\begin{tikzpicture}[scale=0.8]
		\draw[-] (-5,0)  -- (5,0)  node[right] {$CA$}; 
		\draw[-] (0,0)  -- (0,8)  node[left]  {$r_{1}$};
		\draw[smooth, domain = -3.5:3, color=black]
		plot (\x,{3.5+\x}) node[right] {$CA^{0}(r_{1},Q_{1}) $};
		\draw[smooth, domain = -5:2.5, color=black]
		plot (\x,{5.5+\x}) node[right] {$CA^{1}(r_{1},Q_{1}) $};
		\draw[dotted, domain = -5:0, color=black]
		plot (\x,{1}) node[right] {$r^{*} $};
		\draw[dashed, domain = 0:-5, color=black]
		plot (\x,{1-0.5*\x}) node[right] {$r^{*}+\rho (-CA)  $};
		\draw[dotted, domain = 1.83333:0, color=black]
		plot (-1.667,{\x}); 
		\node[below] at (-1.667,0) {$CA_{1}^{0}$};
		\draw[dotted, domain = 2.5:0, color=black]
		plot (-3,{\x}); 
		\node[below] at (-3,0) {$CA_{1}^{1}$};
		\end{tikzpicture}
	\end{center}
\end{frame}

\begin{frame}[label=13]
	\frametitle{{\normalsize AJUSTE EXTERNO ECONOMÍA GRANDE Y ABIERTA} {}}
	En una economía grande y abierta supondremos que:
	\begin{align}
		CA^{US}+CA^{RW}&=0
	\end{align}
	En el gráfico siguiente se muestra el marco analítico de la determinación de la CA es de notar que para el caso de US este se mide de izquierda (-) a derecha (+) lo opuesto para el resto del mundo.
	La conclusión es que los cambios en la CA no son tan pronunciado como en el caso de una economía pequeña.
	
\end{frame}

\begin{frame}[label=8]
	\frametitle{{\normalsize AJUSTE EXTERNO ECONOMÍA GRANDE Y ABIERTA} {}}
	\begin{center}
		\begin{tikzpicture}[scale=0.8]
		\draw[-] (-5,0)  node[left] {$CA^{RW}$} -- (5,0) node[right] {$CA^{US}$}; 
		\draw[-] (0,0)  -- (0,8)  node[left]  {$r_{1}$};
		\draw[smooth, domain = -3.5:3, color=black]
		plot (\x,{3.5+\x}) node[right] {$CA^{US} $};
		\draw[smooth, domain = -5:2.5, color=black]
		plot (\x,{5.5+\x}) node[right] {$CA^{US'} $};
		\draw[smooth, domain = 0.5:-4, color=black]
		plot (\x,{1-1.5*\x}) node[right] {$ CA^{RW}$};
		\draw[fill] (-1.8,3.7) circle [radius=2.5pt]	node[above] {$B$};
		\draw[dotted, domain = 3.7:0, color=black]
		plot (-1.8,{\x}); 
		\draw[dotted, domain = -1.8:0, color=black]
		plot ({\x},3.7); 
		\draw[fill] (-1.0,2.5) circle [radius=2.5pt]	node[above] {$A$};
		\draw[dotted, domain = 2.5:0, color=black]
		plot (-1.0,{\x});
		\draw[fill] (-3.0,2.5) circle [radius=2.5pt]	node[above] {$C$};
		\draw[dotted, domain = 2.5:0, color=black]
		plot (-3.0,{\x});
		\draw[dotted, domain = -3.0:0, color=black]
		plot ({\x},2.5); 
		\draw[fill] (0,5.5) circle [radius=2.5pt]	node[above] {$D^{'}$};
		\draw[fill] (0,3.5) circle [radius=2.5pt]	node[above] {$D$};
		\end{tikzpicture}
	\end{center}
\end{frame}

\begin{frame}[label=13]
	\frametitle{{\normalsize HIPÓTESIS DE LA SUPER-ABUNDANCIA DE AHORRO} {}}
	Un hecho estilizado: US incrementó su déficit en CA entre 1995 a 2005 de 1.5\% al 6.0\% de su PIB. Luego de la gran crisis del 2007 su deficit cayó llegando al 3.0\% de su PIB. \\
	Como explicar ese hecho estilizado se debe a factores externos o internos.\\
	\cite{Bernanke2005} sostiene que es por factores externos; el resto del mundo tiene un exceso de ahorro sin incentivos a invertir ahí sino en US y por consecuencia US debe presentar un déficit de CA.\\
	Específicamente Bernanke atribuye ese exceso de ahorro a:
	\begin{enumerate}
		\item Incremento en la acumulación de reservas internacionales en economía emergente como respuesta precautoria de las crisis de los 90.
		\item Depreciación de las monedas del RW como políticas para impulsar sus exportaciones.
	\end{enumerate}
		
\end{frame} 

\begin{frame}[label=14]
	\frametitle{{\normalsize HIPÓTESIS HECHO EN ESTADOS UNIDOS} {}}
	Ante el mismo hecho estilizado hay otra hipótesis: Hecho en USA un resultado de la evolución económica de los USA.\\
	 
	Para el análisis de ambas hipótesis el marco analítico gráfico es útil. Es de notar que la diferencia en ambos hipótesis radica en el comportamiento de la tasa de interés.   
	
\end{frame} 

\begin{frame}[label=9]
	\frametitle{{\normalsize HIPÓTESIS DE LA SUPER-ABUNDANCIA DE AHORRO} {}}
	\begin{center}
		\begin{tikzpicture}[scale=0.8]
		\draw[-] (-5,0)  node[left] {$CA^{RW}$} -- (5,0) node[right] {$CA^{US}$}; 
		\draw[-] (0,0)  -- (0,8)  node[left]  {$r_{1}$};
		\draw[smooth, domain = -3.5:3, color=black]
		plot (\x,{3.5+\x}) node[above] {$CA^{US}(r) $};
		\draw[smooth, domain = 0.5:-4, color=black]
		plot (\x,{1-1.5*\x}) node[above] {$ CA^{RW'}(r)$};
		\draw[smooth, domain = 3:-2.0, color=black]
		plot (\x,{5-1.5*\x}) node[above] {$ CA^{RW}(r)$};
		\draw[fill] (0.6,4.1) circle [radius=2.5pt]	node[above] {$A$};
		\draw[dotted, domain = 4.1:0, color=black]
		plot (0.6,{\x}) node[below] {$ CA^{US^{0}}$}; 
		\draw[dotted, domain = 0.6:0, color=black]
		plot ({\x},4.1) node[left] {$ r*^{0}$}; 
		\draw[fill] (-1.0,2.5) circle [radius=2.5pt]node[above] {$B$};
		\draw[dotted, domain = 2.5:0, color=black]
		plot (-1.0,{\x}) node[below] {$ CA^{US^{1}}$}; 
		\draw[dotted, domain = -1.0:0, color=black]
		plot ({\x},2.5) node[right] {$ r*^{1}$};
		\draw[->, blue, ultra thick] (-1,6.5)  -- (-2.5,5) ; 
		\end{tikzpicture}
	\end{center}
\end{frame}

\begin{frame}[label=10]
	\frametitle{{\normalsize HIPÓTESIS HECHO EN ESTADOS UNIDOS} {}}
	\begin{center}
		\begin{tikzpicture}[scale=0.8]
		\draw[-] (-5,0)  node[left] {$CA^{RW}$} -- (5,0) node[right] {$CA^{US}$}; 
		\draw[-] (0,0)  -- (0,8)  node[left]  {$r_{1}$};
		\draw[smooth, domain = -3.5:3, color=black]
		plot (\x,{3.5+\x}) node[above] {$CA^{US}(r) $};
		\draw[smooth, domain = -5.5:1.0, color=black]
		plot (\x,{7+\x}) node[above] {$CA^{US'}(r) $};
		\draw[smooth, domain = 3:-2.0, color=black]
		plot (\x,{5-1.5*\x}) node[above] {$ CA^{RW}(r)$};
		\draw[fill] (0.6,4.1) circle [radius=2.5pt]	node[above] {$A$};
		\draw[dotted, domain = 4.1:0, color=black]
		plot (0.6,{\x}) node[below] {$ CA^{US^{0}}$}; 
		\draw[dotted, domain = 0.6:0, color=black]
		plot ({\x},4.1) node[left] {$ r*^{0}$}; 
		\draw[fill] (-1.33,5.666) circle [radius=2.5pt]	node[above] {$B$};
		\draw[dotted, domain = 5.666:0, color=black]
		plot (-1.33,{\x}) node[below] {$ CA^{US^{1}}$}; 
		\draw[dotted, domain = -1.33:0, color=black]
		plot ({\x},5.66) node[right] {$ r*^{1}$};
		\draw[->, blue, ultra thick] (-3,0.5)  -- (-4.25,2.375) ; 
		\end{tikzpicture}
	\end{center}
\end{frame}
\begin{frame}[label=15]
	\frametitle{{\normalsize HIPÓTESIS DE LA SUPER-ABUNDANCIA DE AHORRO} {}}
	Sí bien la Hipótesis de la super-abundancia de ahorro parece explicar en el comportamiento de la CA antes de la crisis, ¿cómo se explica después de la crisis?\\
	
	
	 
	
\end{frame} 

	\begin{frame}
		\frametitle{{\large 
				Bibliografía}}
		\renewcommand{\refname}{Referencias}
		\bibliography{Biblioteca}
		\bibliographystyle{flexbib}
	\end{frame}


\end{document}
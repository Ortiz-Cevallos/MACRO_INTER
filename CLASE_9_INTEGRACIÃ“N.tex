\documentclass[10pt, xcolor=table, x11names]{beamer}
\usepackage[spanish]{babel} %CORTE DE PALABRAS RESPETANDO EL IDIOMA ESPAÑOL.
\usepackage[Utf8]{inputenc} %acentos desde el teclado
\usepackage	{textpos}
\usefonttheme{professionalfonts} % fuentes de LaTeX
\usetheme{Boadilla}      % or try Darmstadt, Madrid, Warsaw, ...
\usecolortheme[RGB={130,130,190}]{structure} % or try albatross, beaver, crane, ...
\useinnertheme{rounded}
%\useoutertheme{shadow}
\setbeamertemplate{blocks}[rounded][shadow=true]
\setbeamertemplate{navigation symbols}{}
\setbeamercovered{transparent} % Velos
\setbeamertemplate{caption}[numbered]
%\usepackage[spanish, authoryear, roud, datebegin]{flexbib} %CITAS BIBLIOGRÁFICAS
\newtheorem{Teorema}{Teorema}
\usepackage{ragged2e}
\justifying
\usepackage{booktabs}
\usepackage{multirow}
\usepackage[x11names,table]{xcolor}
%\usepackage[pdftex]{graphicx}
\usepackage{epstopdf} % Convertir .eps a .pdf (si fuera necesario)
\DeclareGraphicsExtensions{.pdf,.png,.jpg, .eps} % busca en este orden!
\author[Luis Ortiz Cevallos e-mail: \href{leortiz@uc.cl}{\textit{leortiz@uc.cl}}]{Profesor: Luis Ortiz Cevallos, e-mail:\href{leortiz@uc.cl}{\textit{leortiz@uc.cl}} }
\title[MACRO INTERNACIONAL]{\vspace*{1.0em} MACROECONOMÍA INTERNACIONAL}
\date[\href{https://ortiz-cevallos.github.io/luisortiz.github.io/ }{\textit{https://ortiz-cevallos.github.io/luisortiz.github.io/}}]{}
%\usepackage[pdftex]{hyperref}
\usepackage{tikz}
%\usepackage{pstricks}
\hypersetup{colorlinks,%
	citecolor=blue,%
	filecolor=blue,%
	linkcolor=blue,%
	urlcolor=blue,%
	pdftex}

\begin{document}


\begin{frame}
\titlepage
\end{frame}


\begin{frame}[label=1]
	\frametitle{{\normalsize INTEGRACIÓN A LOS MERCADOS DE CAPITALES: ANTECEDENTES } {}}
	Desde los 70 una serie de eventos han hecho el supuesto de libre movilidad de mercados de capitales una realidad:
	
	\begin{itemize}
		\item El quiebre de Bretton-Woods y la consecuencias de remoción en controles de capitales en algunas economías de Europa como la Alemana.
		\item La alta inflación de USA junto a la aplicación de la regulación financiera Q, que provocó un rápido crecimiento en el mercado de moneda europea (eurodollar).
		\item Avances tecnológicos en los procesos de información.
		\item Desregulación de los mercados en los 80.
		\item La unificación económica y monetaria de Europa.
	\end{itemize}
\end{frame}

\begin{frame}[label=2]
	\frametitle{{\normalsize MIDIENDO EL GRADO DE MOVILIDAD DE CAPITAL: F-H HIPÓTESIS } {}}
En \cite{Feldstein79} con una muestra de 16 países para el período 1960-1974 para el promedio del ratio ahorro a producto ($\frac{S_{i}}{PIB_{i}}$)  obtienen la siguiente regresión: 

\begin{align}
\frac{I_{i}}{PIB_{i}}&=0.035+0.887\frac{S_{i}}{PIB_{i}} \nonumber
\end{align}

Su hipótesis es que en presencia de libre movilidad de capitales la relación entre la inversión y el ahorro debería ser cercana a cero y por tanto sus resultados es evidencia de la baja movilidad de capitales.

\end{frame}


\begin{frame}[plain, label=3]
	\frametitle{{\normalsize MIDIENDO EL GRADO DE MOVILIDAD DE CAPITAL: F-H HIPÓTESIS } {}}
	La justificación de F-H  HIPÓTESIS viene dado por la identidad:
	\begin{align}
	CA&=S-I \nonumber
	\end{align}
	\begin{figure}[H]
	\renewcommand{\figurename}{Figura}
	\caption{\sc Respuesta de S y I a cambios independientes}
	\begin{minipage}[t]{0.48\textwidth}
		\begin{center}
			\begin{tikzpicture}[scale=0.45]
			\draw[-] (0,0)  -- (8,0)  node[right] {{\tiny$S,I$}}; 
			\draw[-] (0,0)  -- (0,8)  node[left]  {{\tiny$r_{1}$}};
			\draw[smooth, domain = 6.5:0.25, color=black]
			plot (\x,{7-\x}) node[right] {{\tiny$I(r_{1}) $}};
			\draw[smooth, domain = 0.75:6.75, color=black]
			plot (\x,{0.75+\x}) node[right] {{\tiny$S^{0}(r_{1}) $}};
			\draw[smooth, domain = 1.75:6.75, color=blue]
			plot (\x,{-1.5+\x}) node[right] {{\tiny$S^{1}(r_{1}) $}};
			\draw[dotted, domain = 0:8, color=black]
			plot (\x,{2}); 
			\node[left] at (0,2) {$r^{*}$};
			\draw[dotted, domain = 2:0, color=black]
			plot (1.25,{\x}); 
			\node[below] at (1.25,0) {{\tiny$S^{0}$}};
			\draw[dotted, domain = 2:0, color=black]
			plot (3.5,{\x}); 
			\node[below] at (3.5,0) {{\tiny$S^{1}$}};
			\draw[dotted, domain = 2:0, color=black]
			plot (5,{\x}); 
			\node[below] at (5,0) {{\tiny$I^{0}$}};
			\end{tikzpicture}
		\end{center}		
	\end{minipage} \hfill \begin{minipage}[t]{0.48\textwidth}
	\begin{center}
		\begin{tikzpicture}[scale=0.45]
		\draw[-] (0,0)  -- (8,0)  node[right] {{\tiny$S,I$}}; 
		\draw[-] (0,0)  -- (0,8)  node[left]  {{\tiny$r_{1}$}};
		\draw[smooth, domain = 6.5:0.25, color=black]
		plot (\x,{7-\x}) node[right] {{\tiny$I^{0}(r_{1}) $}};
		\draw[smooth, domain = 0.75:6.75, color=black]
		plot (\x,{0.75+\x}) node[right] {{\tiny$S(r_{1}) $}};
		\draw[smooth, domain = 8.5:1, color=blue]
		plot (\x,{9-\x}) node[right] {{\tiny$I^{1}(r_{1}) $}};
		\draw[dotted, domain = 0:8, color=black]
		plot (\x,{2}); 
		\node[left] at (0,2) {$r^{*}$};
		\draw[dotted, domain = 2:0, color=black]
		plot (1.25,{\x}); 
		\node[below] at (1.25,0) {{\tiny$S^{0}$}};
		\draw[dotted, domain = 2:0, color=black]
		plot (7,{\x}); 
		\node[below] at (7,0) {{\tiny$I^{1}$}};
		\draw[dotted, domain = 2:0, color=black]
		plot (5,{\x}); 
		\node[below] at (5,0) {{\tiny$I^{0}$}};
		\end{tikzpicture}
	\end{center}
\end{minipage}
\label{IV_G4}
\end{figure}


	Noten que la economía sin libre movilidad de capitales la CA es siempre cero y por tanto $S=I$; en contraste para una economía pequeña y abierta la tasa de interés es exógena y por tanto si el ahorro y la inversión es afectada por factores independiente su correlación es cero.
	
	
\end{frame}

\begin{frame}[plain,label=4]
	\frametitle{{\normalsize F-H HIPÓTESIS: Críticas } {}}
	La primera crítica es que aún en presencia de libre movilidad de mercados de capitales, pueden haber una fuerte asociación entre ahorro e inversión dado que ambos pueden verse afectados por factores idénticos. 
	\begin{block}{Ejemplo}
		Dada las funciones de producción para los períodos 1 y 2:
		\begin{align}
		Q_{1}&=A_{1}F(k_{1})\nonumber\\
		Q_{2}&=A_{2}F(k_{2})\nonumber
		\end{align}
		Ahora asuma que $A_{1}$ se ve afectado por un choque positivo $\epsilon_{1}$ el cual determina el choque en el período 2 bajo el proceso $\epsilon_{2}=\rho\epsilon_{1} $ con $ 0<\rho<1$
		Entonces:
		\begin{align}
		E_{1}A_{2}\uparrow\rightarrow k_{2}\uparrow\rightarrow I_{1}\uparrow \nonumber\\
		E_{1}A_{2}\uparrow\rightarrow S_{1}\downarrow\nonumber\\
		A_{1}\uparrow\rightarrow Q_{1}\uparrow\rightarrow S_{1}\uparrow\nonumber\\
		\dfrac{\delta S_{1}}{\delta A_{1}}>	\dfrac{\delta S_{1}}{\delta E_{1}A_{2}}\nonumber	
		\end{align}
	\end{block}
	
\end{frame}

\begin{frame}[plain,label=5]
	\frametitle{{\normalsize F-H HIPÓTESIS: Críticas } {}}
	La segunda crítica es que aún en presencia de libre movilidad de mercados de capitales, pueden haber una fuerte asociación entre ahorro e inversión si se trata de una economía grande. 

\begin{figure}[H]
	\renewcommand{\figurename}{Figura}
	\caption{\sc Ahorro, Inversión y Cuenta Corriente}
	\begin{minipage}[t]{0.48\textwidth}
		\begin{center}
			\begin{tikzpicture}[scale=0.48]
			\draw[-] (0,0)  -- (8,0)  node[right] {{\tiny$S,I$}}; 
			\draw[-] (0,0)  -- (0,8)  node[left]  {{\tiny$r$}};
			\draw[smooth, domain = 0.75:7.5, color=black]
			plot (\x,{8.5-\x}) node[right] {{\tiny$I(r) $}};
			\draw[smooth, domain = 0.25:7, color=black]
			plot (\x,{0+\x}) node[right] {{\tiny$S(r) $}};
			\draw[smooth, domain = 2.25:8, color=blue]
			plot (\x,{-2+\x}) node[right] {{\tiny$S^{'}(r) $}};
			\draw[dotted, domain = 0:7, color=black]
			plot (\x,{2.5}); 
			\node[left] at (0,2.5) {$r^{*1}$};
			\draw[dotted, domain = 0:7, color=black]
			plot (\x,{3.5}); 
			\node[left] at (0,3.5) {$r^{*}$};
			\draw[dotted, domain = 2.5:0, color=black]
			plot (3.5,{\x}); 
			\node[below] at (3.5,0) {{\tiny$S$}};
			\draw[dotted, domain = 2.5:0, color=black]
			plot (4.5,{\x}); 
			\node[below] at (4.5,0) {{\tiny$S^{'}$}};
			\draw[dotted, domain = 3.5:0, color=black]
			plot (5,{\x}); 
			\node[below] at (5,0) {{\tiny$I$}};
			\draw[dotted, domain = 2.5:0, color=black]
			plot (6,{\x}); 
			\node[below] at (6,0) {{\tiny$I^{'}$}};
			\end{tikzpicture}
		\end{center}		
	\end{minipage} \hfill \begin{minipage}[t]{0.48\textwidth}
	\begin{center}
		\begin{tikzpicture}[scale=0.48]
		\draw[-] (-4,0)  -- (4,0)  node[right] {{\tiny $CA$}}; 
		\draw[-] (0,0)  -- (0,8)  node[left]  {{\tiny$r$}};
		\draw[smooth, domain = -3:4, color=blue]
		plot (\x,{3.5+\x}) node[right] {{\tiny $CA^{'}(r) $}};
		\draw[smooth, domain = -4:3.0, color=black]
		plot (\x,{5.17+\x}) node[right] {{\tiny $CA(r) $}};
		\draw[dotted, domain = -4:0, color=black]
		plot (\x,{2.5}); 
		\node[right] at (0,2.5) {$r^{*1}$};
		\draw[dotted, domain = -4:0, color=black]
		plot (\x,{3.5}); 
		\node[right] at (0,3.5) {$r^{*}$};
		\draw[smooth, domain = 0:-4, color=black]
		plot (\x,{1-1.5*\x}) node[right] {{\tiny$CA^{RW}(r) $}};
		\node[below] at (-4,0) {{\tiny$CA^{RW}$}};
		\end{tikzpicture}
	\end{center}
\end{minipage}
\label{IV_G2}
\end{figure}

	
\end{frame}


\end{document}
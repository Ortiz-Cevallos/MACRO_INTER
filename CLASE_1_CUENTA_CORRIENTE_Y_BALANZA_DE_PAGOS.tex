\documentclass[10pt, xcolor=table, x11names]{beamer}
\usepackage[spanish]{babel} %CORTE DE PALABRAS RESPETANDO EL IDIOMA ESPAÑOL.
\usepackage[Utf8]{inputenc} %acentos desde el teclado
\usepackage	{textpos}
\usefonttheme{professionalfonts} % fuentes de LaTeX
\usetheme{Boadilla}      % or try Darmstadt, Madrid, Warsaw, ...
\usecolortheme[RGB={130,130,190}]{structure} % or try albatross, beaver, crane, ...
\useinnertheme{rounded}
%\useoutertheme{shadow}
\setbeamertemplate{blocks}[rounded][shadow=true]
\setbeamertemplate{navigation symbols}{}
\setbeamercovered{transparent} % Velos
\setbeamertemplate{caption}[numbered]
%\usepackage[spanish, authoryear, roud, datebegin]{flexbib} %CITAS BIBLIOGRÁFICAS
\newtheorem{Teorema}{Teorema}
\usepackage{ragged2e}
\justifying
\usepackage{booktabs}
\usepackage{multirow}
\usepackage[x11names,table]{xcolor}
\usepackage[pdftex]{graphicx}
\usepackage{epstopdf} % Convertir .eps a .pdf (si fuera necesario)
\DeclareGraphicsExtensions{.pdf,.png,.jpg, .eps} % busca en este orden!
\author[Luis Ortiz Cevallos e-mail: \href{leortiz@uc.cl}{\textit{leortiz@uc.cl}}]{Profesor: Luis Ortiz Cevallos, e-mail:\href{leortiz@uc.cl}{\textit{leortiz@uc.cl}} }
\title[MACRO INTERNACIONAL]{\vspace*{1.0em} MACROECONOMÍA INTERNACIONAL}
\date[\href{https://ortiz-cevallos.github.io/luisortiz.github.io/ }{\textit{https://ortiz-cevallos.github.io/luisortiz.github.io/}}]{}

\usepackage[pdftex]{hyperref}
\hypersetup{colorlinks,%
	citecolor=blue,%
	filecolor=blue,%
	linkcolor=blue,%
	urlcolor=blue,%
	pdftex}

\begin{document}


\begin{frame}
\titlepage
\end{frame}




\begin{frame}[label=INTRODUCCION]
	\frametitle{{\normalsize INTRODUCCIÓN} {}}
	La balanza de pagos de un país se constituye por dos componentes:
	\begin{enumerate}
		\item Cuenta Corriente: Recoge el flujo de exportaciones netas de bienes y servicios junto el pago internacional neto de ingresos.
		\item Cuenta Financiera: Recoge las ventas de activos nacionales a los extranjeros y las compras de activos del extranjero.
	\end{enumerate}
	La identidad fundamental de la balanza de pagos es:
	\begin{align}
	CC_{t}&=-CF_{t}
	\end{align}
	Donde $CC_{t}$ es la Cuenta Corriente en t y $CF_{t}$ es la Cuenta Financiera en t.
\end{frame}

\begin{frame}[label=Convenciones Contables]
	\frametitle{{\normalsize Convenciones Contables} {}}
		Algunas convenciones para el registro de transacciones de la Cuenta Corriente:
		\begin{enumerate}
			\item Las exportaciones y los ingresos recibidos del exterior son registrados con el signo positivo.
			\item Las importaciones como los ingresos pagados al exterior son registrados con el signo negativo.
		\end{enumerate}
			Algunas convenciones para el registro de transacciones de la Cuenta Financiera:
			\begin{enumerate}
				\item La venta de un activos al extranjero (Adquisición de un pasivo con el extranjero) es registrado con el signo positivo.
				\item La compra de un activos del extranjero (Disminución de un pasivo del extranjero) es registrado con el signo negativo. 
			\end{enumerate}
	\end{frame}
	
	\begin{frame}[label=Cuentas de BP]
		\frametitle{{\normalsize Cuentas de BP} {}}
		\begin{enumerate}
			\item Cuenta Corriente.
			\begin{enumerate}
				\item Balanza Comercial
				\begin{enumerate}
					\item Exportaciones de bienes
					\item Importaciones de bienes
				\end{enumerate}
				\item Balanza de servicios
				\begin{enumerate}
					\item Servicios no factoriales (fletes, seguros, turismos etc.)
					\item Servicios de Capital (pago de interés, remesas de utilidades)
					\item Servicios laborales (pagos de salarios)
				\end{enumerate}
			\end{enumerate}	
				\item Cuentas de Capitales
				\begin{enumerate}
					\item Inversión extranjera neta recibida
					\item Créditos extranjeros netos recibidos
					\begin{enumerate}
						\item Corto plazo
						\item Largo plazo
					\end{enumerate}
				\end{enumerate}	
				\item Errores y omisiones 
				\item Resultado de Balanza de pagos (=Variación en las Reservas Internacionales oficiales netas)
			\end{enumerate}
	\end{frame}
	
	
	\begin{frame}[label=Cuenta Corriente y PNII]
		\frametitle{{\normalsize Cuenta Corriente y PII} {}}
			Una razón por que el concepto de Cuenta Corriente es importante es por que refleja las necesidades de financiamiento externo de los países.\\
			Por ejemplo sí un país cuenta con déficit en CC implica que debe vender activos nacionales o adquirir pasivos con el extranjero, lo que se conoce como una entrada de capitales.\\
			La Posición Inversiones Internacionales (PII) representa la riqueza extranjera en propiedad nacional y es la diferencia del valor de los activos extranjeros en propiedad del país respecto al valor de los activos naciones en propiedad de los extranjeros.\\
			La PII puede afectarse de dos maneras:
			\begin{enumerate}
				\item Por cambios en la cuenta corriente, por ejemplo un déficit implica un cambio negativo en PII (Se venden activos extranjeros o bien se adquieren pasivos con el extranjero).
				\item Cambio en el precio de los activos extranjeros y nacionales (ver \cite{Hausmann2006} y \cite{Milesi2008}).
			\end{enumerate}
		\end{frame}
		
	\begin{frame}
		\frametitle{{\large 
				Bibliografía}}
		\renewcommand{\refname}{Referencias}
		\bibliography{Biblioteca}
		\bibliographystyle{flexbib}
	\end{frame}
	
	
	
	
	
	
\end{document}
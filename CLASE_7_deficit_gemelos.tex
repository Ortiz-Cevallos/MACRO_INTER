\documentclass[10pt, xcolor=table, x11names]{beamer}
\usepackage[spanish]{babel} %CORTE DE PALABRAS RESPETANDO EL IDIOMA ESPAÑOL.
\usepackage[Utf8]{inputenc} %acentos desde el teclado
\usepackage	{textpos}
\usefonttheme{professionalfonts} % fuentes de LaTeX
\usetheme{Boadilla}      % or try Darmstadt, Madrid, Warsaw, ...
\usecolortheme[RGB={130,130,190}]{structure} % or try albatross, beaver, crane, ...
\useinnertheme{rounded}
%\useoutertheme{shadow}
\setbeamertemplate{blocks}[rounded][shadow=true]
\setbeamertemplate{navigation symbols}{}
\setbeamercovered{transparent} % Velos
\setbeamertemplate{caption}[numbered]
%\usepackage[spanish, authoryear, roud, datebegin]{flexbib} %CITAS BIBLIOGRÁFICAS
\newtheorem{Teorema}{Teorema}
\usepackage{ragged2e}
\justifying
\usepackage{booktabs}
\usepackage{multirow}
\usepackage[x11names,table]{xcolor}
%\usepackage[pdftex]{graphicx}
\usepackage{epstopdf} % Convertir .eps a .pdf (si fuera necesario)
\DeclareGraphicsExtensions{.pdf,.png,.jpg, .eps} % busca en este orden!
\author[Luis Ortiz Cevallos e-mail: \href{leortiz@uc.cl}{\textit{leortiz@uc.cl}}]{Profesor: Luis Ortiz Cevallos, e-mail:\href{leortiz@uc.cl}{\textit{leortiz@uc.cl}} }
\title[MACRO INTERNACIONAL]{\vspace*{1.0em} MACROECONOMÍA INTERNACIONAL}
\date[\href{https://ortiz-cevallos.github.io/luisortiz.github.io/ }{\textit{https://ortiz-cevallos.github.io/luisortiz.github.io/}}]{}
%\usepackage[pdftex]{hyperref}
\usepackage{tikz}
%\usepackage{pstricks}
\hypersetup{colorlinks,%
	citecolor=blue,%
	filecolor=blue,%
	linkcolor=blue,%
	urlcolor=blue,%
	pdftex}

\begin{document}


\begin{frame}
\titlepage
\end{frame}


\begin{frame}[label=1]
	\frametitle{{\normalsize DÉFICIT GEMELOS} {}}
	\begin{block} {Motivación}
		Hasta ahora sólo se consideran dos agentes: hogares y firmas.\\
		Se debe introducir el gobierno pues éste influye a través de impuestos, transferencias, consumo e inversión. El objetivo es conocer cual es el rol del gobierno en la determinación de la cuenta corriente. 
	\end{block}	
	\begin{block}{Hipotesis de los Déficits Gemelos}
		Esta hipótesis consiste en que el déficit fiscal es el que conduce el déficit en cuenta corriente. A groso modo:
		\begin{align}
		CA&=S-I\nonumber \\
		S&=S_{priv}+S_{pub}\nonumber \\
		I&=I_{priv}+I_{pub}\nonumber \\
		S_{pub}&=Ingresos_{pub}-C_{pub}
		\end{align}
		Entonces un aumento del déficit fiscal (reducción del ahorro público un mayor gasto público) implica un déficit en cuenta corriente.
	\end{block} 
\end{frame}

\begin{frame}[label=2]
	\frametitle{{\normalsize Evidencia Empírica: DÉFICIT GEMELOS} {}}
	Considerando cuatro eventos para evaluar la hipótesis de los déficit gemelos:
	\begin{itemize}
		\item EEUU. principios de los 80 durante el período de Reagan. Aparentemente si se cumplió la hipótesis.
		\item EEUU. 2007-2008 durante la expansión fiscal de Obama. Aparentemente no se cumplió la hipótesis.
		\item EEUU. II-WW. Aparentemente si se cumplió la hipótesis en la dirección pero en magnitud pequeña.
		\item EEUU. II-WW. Aparentemente si se cumplió la hipótesis en la dirección pero en magnitud pequeña.
		\item EEUU. 90-2000 administración de Clinton. Aparentemente no se cumplió la hipótesis. 
		
		\end{itemize}
	
\end{frame}

\begin{frame}[label=3]
	\frametitle{{\normalsize Evidencia Empírica: DEFICIT GEMELOS} {}}
	El que no exista una relación sistemática entre grandes cambios en el ahorro fiscal con deterioros en la cuenta corriente no significa que la hipótesis de déficit gemelos no se cumpla. \\
	
	Toda economía enfrenta diferentes shock por lo que es difícil aislar los efectos de una variable ante cambios en otra. 
	
	Veamos un ejemplo: El incremento del déficit en cuenta corriente en Estados unidos en los primeros años de la década de los 80, era motivada por:\\☺
	 {\bf Hipotesis 1: Mayor demanda del resto del mundo de activos de Estados Unidos:} 
	 	\begin{itemize}
		\item ``Safe heaven''y  Estados Unidos recibió ``capital flight'' de América Latina.
		\item La crisis de la deuda.
		\item Desregulación financiera en múltiples países.
	\end{itemize}
		
\end{frame}

\begin{frame}[label=4]
	\frametitle{{\normalsize Evidencia Empírica: DEFICIT GEMELOS} {}}
		\begin{center}
			\begin{tikzpicture}[scale=0.8]
			\draw[-] (-5,0)  node[left] {$CA^{RW}$} -- (5,0) node[right] {$CA^{US}$}; 
			\draw[-] (0,0)  -- (0,8)  node[left]  {$r_{1}$};
			\draw[smooth, domain = -3.5:3, color=black]
			plot (\x,{3.5+\x}) node[above] {$CA^{US}(r) $};
			\draw[smooth, domain = 0.5:-4, color=black]
			plot (\x,{1-1.5*\x}) node[above] {$ CA^{RW'}(r)$};
			\draw[smooth, domain = 3:-2.0, color=black]
			plot (\x,{5-1.5*\x}) node[above] {$ CA^{RW}(r)$};
			\draw[fill] (0.6,4.1) circle [radius=2.5pt]	node[above] {$A$};
			\draw[dotted, domain = 4.1:0, color=black]
			plot (0.6,{\x}) node[below] {$ CA^{US^{0}}$}; 
			\draw[dotted, domain = 0.6:0, color=black]
			plot ({\x},4.1) node[left] {$ r*^{0}$}; 
			\draw[fill] (-1.0,2.5) circle [radius=2.5pt]node[above] {$B$};
			\draw[dotted, domain = 2.5:0, color=black]
			plot (-1.0,{\x}) node[below] {$ CA^{US^{1}}$}; 
			\draw[dotted, domain = -1.0:0, color=black]
			plot ({\x},2.5) node[right] {$ r*^{1}$};
			\draw[->, blue, ultra thick] (-1,6.5)  -- (-2.5,5) ; 
			\end{tikzpicture}
		\end{center}
\end{frame}

\begin{frame}[label=5]
	\frametitle{{\normalsize Evidencia Empírica: DEFICIT GEMELOS} {}}
	El incremento del déficit en cuenta corriente en Estados unidos en los primeros años de la década de los 80, era motivada por:\\☺
		{\bf Hipótesis 2: En EEUU los agentes desean a cualquier nivel de interés ahorrar menos que antes} 
	
\end{frame}


\begin{frame}[label=6]
	\frametitle{{\normalsize Evidencia Empírica: DEFICIT GEMELOS} {}}
\begin{center}
	\begin{tikzpicture}[scale=0.8]
	\draw[-] (-5,0)  node[left] {$CA^{RW}$} -- (5,0) node[right] {$CA^{US}$}; 
	\draw[-] (0,0)  -- (0,8)  node[left]  {$r_{1}$};
	\draw[smooth, domain = -3.5:3, color=black]
	plot (\x,{3.5+\x}) node[above] {$CA^{US}(r) $};
	\draw[smooth, domain = -5.5:1.0, color=black]
	plot (\x,{7+\x}) node[above] {$CA^{US'}(r) $};
	\draw[smooth, domain = 3:-2.0, color=black]
	plot (\x,{5-1.5*\x}) node[above] {$ CA^{RW}(r)$};
	\draw[fill] (0.6,4.1) circle [radius=2.5pt]	node[above] {$A$};
	\draw[dotted, domain = 4.1:0, color=black]
	plot (0.6,{\x}) node[below] {$ CA^{US^{0}}$}; 
	\draw[dotted, domain = 0.6:0, color=black]
	plot ({\x},4.1) node[left] {$ r*^{0}$}; 
	\draw[fill] (-1.33,5.666) circle [radius=2.5pt]	node[above] {$B$};
	\draw[dotted, domain = 5.666:0, color=black]
	plot (-1.33,{\x}) node[below] {$ CA^{US^{1}}$}; 
	\draw[dotted, domain = -1.33:0, color=black]
	plot ({\x},5.66) node[right] {$ r*^{1}$};
	\draw[->, blue, ultra thick] (-3,0.5)  -- (-4.25,2.375) ; 
	\end{tikzpicture}
\end{center}
\end{frame}

\begin{frame}[label=7]
	\frametitle{{\normalsize Evidencia Empírica: DEFICIT GEMELOS} {}}
	Entre las dos hipótesis anteriores como elegir la correcta.\\
	Una estrategia es ver entre las variables económicas involucradas cuales tiene según hipótesis diferente predicción y comparar con los datos observados.\\
	Entonces la variable candidata es la tasa de interés real; según la hipótesis 1 ésta cae, dado que en la economía hay más ahorro; y según la hipótesis 2 ésta sube por que la demanda de ahorro aumenta.
	La tasa de interés real observada favorece a la hipótesis 2. Lo que ocurrió es que hubo un desplazamiento de la inversión hacia la derecha y/o un desplazamiento del ahorro hacia la izquierda.
\end{frame}


\begin{frame}[label=8]
	\frametitle{{\normalsize Evidencia Empírica: DEFICIT GEMELOS} {}}
	\begin{center}
		\begin{tikzpicture}[scale=0.8]
		\draw[-] (0,0)  -- (5,0) node[right] {$S, I$}; 
		\draw[-] (0,0)  -- (0,6)  node[left]  {$r_{1}$};
		\draw[smooth, domain = 0.2:4, color=black]
		plot (\x,{1+\x}) node[above] {$S$};
		\draw[smooth, domain = 0.5:4, color=black]
		plot (\x,{2+\x}) node[above] {$S^{'}$};
		\draw[smooth, domain = 0.2:4, color=black]
		plot (\x,{5-\x}) node[above] {$I$};
		\draw[smooth, domain = 0.2:4, color=black]
		plot (\x,{5-\x}) node[above] {$I$};
		\draw[smooth, domain = 0.5:5, color=black]
		plot (\x,{6-\x}) node[above] {$I^{'}$};
		\draw[dotted, domain = 4.8:0, color=black]
		plot ({\x},2.7) node[left] {$ r*$}; 
		\end{tikzpicture}
	\end{center}
\end{frame}


\begin{frame}[label=9]
	\frametitle{{\normalsize El sector gobierno en una economía abierta} {}}
	Considerando el modelo de dos períodos, incluyendo el gobierno quien consumo bienes y servicios: $G_{1}, G_{2} $. A la vez el gobierno cobra impuestos: $T_{1}, T_{2} $. Asuma también que  el gobierno en el principio dispone de activos equivalentes a:  $B_{0}^{g} $. Así el gobierno enfrenta las siguientes restricciones en el período 1 y 2:
	\begin{align}
		G_{1}+(B_{1}^{g}-B_{0}^{g})=r_{0}B_{0}^{g}+T_{1}\\
		G_{2}+(B_{2}^{g}-B_{1}^{g})=r_{1}B_{1}^{g}+T_{2}
	\end{align}
	Donde $ B_{1}^{g}$ y $ B_{1}^{g}$ representa el stock de activos del gobierno al final del período 1 y 2 respectivamente.\\
	Al igual que en los hogares prevalecen los principios de transversabilidad y no-ponzi por tanto:
	\begin{align}
	B_{2}^{g}=0
	\end{align}
	Al combinar las 3 ecuaciones tenemos
	\begin{align}
	G_{2}-B_{1}^{g}=r_{1}B_{1}^{g}+T_{2}\nonumber\\
	G_{2}-(1+r_{1})B_{1}^{g}=T_{2}\nonumber\\
	G_{1}+\frac{G_{2}}{1+r_{1}}=(1+r_{0})B_{0}^{g}+T_{1}+\frac{T_{2}}{1+r_{1}}
	\end{align}
\end{frame}

\begin{frame}[label=10]
	\frametitle{{\normalsize El sector gobierno en una economía abierta} {}}
	HOLA
\end{frame}

\end{document}